\documentclass[a4paper,11pt]{article}
\usepackage[left=2.5cm, right=2.5cm, top=1.5cm, bottom=1.5cm]{geometry}
\usepackage{graphicx}
\usepackage{amssymb}
\usepackage{amsmath}
\usepackage[procnames]{listings}
\usepackage{xcolor}
\usepackage[active,tightpage]{preview}
\usepackage{hyperref}

\hypersetup{ %color attributes of citation, link, etc.
    colorlinks=true,
    linkcolor=blue,
    filecolor=gray,
    urlcolor=blue,
    citecolor=blue,
}

\setlength{\parindent}{0pt}

\renewcommand{\PreviewBorder}{1in}
\newcommand{\Newpage}{\end{preview}\begin{preview}}
\newcommand{\matlab}{\textsc{Matlab}} %very important and totally necessary addition
\newcommand{\parallelsum}{\mathbin{\!/\mkern-5mu/\!}}

\newcommand\Item[1][]{%
  \ifx\relax#1\relax  \item \else \item[#1] \fi
  \abovedisplayskip=0pt\abovedisplayshortskip=0pt~\vspace*{-\baselineskip}}

%'codify' text for snippets
\usepackage{xcolor}
\definecolor{codegray}{gray}{1}
\newcommand{\code}[1]{\colorbox{codegray}{\texttt{#1}}}

\definecolor{keywords}{RGB}{255,0,90}
\definecolor{comments}{RGB}{0,0,113}
\definecolor{p_red}{RGB}{160,0,0}
\definecolor{p_green}{RGB}{0,150,0} 
\lstset{language=Python, 
        basicstyle=\ttfamily\small, 
        keywordstyle=\color{keywords},
        commentstyle=\color{comments},
        stringstyle=\color{p_red},
        showstringspaces=false,
        identifierstyle=\color{p_green},
		procnamekeys={def,class}}

\graphicspath{ {./images/} }
           
\begin{document}
\begin{preview}
\title{\LARGE{\textbf{ENGR222 Assignment 2}}}
\author{Niels Clayton : 300437590}
\date{}
\maketitle
\hrule

\begin{enumerate}
    \item The following questions are concerned with the function
    $$ f(x,y) = -2x^3 + 3x^2y + 2y^3 - 9y +5$$

    \begin{enumerate}
        \item Determine the first order partial derivative of $f(x, y)$
        \begin{align*}
            f_x &= -6x^2 + 6xy \\
            f_y &=  6y^2 + 3x^2 - 9
        \end{align*}
    
        \item Determine the second order partial derivatives of $f(x, y)$
        \begin{align*}
            f_{xx} &= -12x + 6y\\
            f_{yy} &= 12y \\
            f_{xy} &= 6x\\
        \end{align*}
    
        \item Find all of the critical points of $f(x, y)\\$

            $\mathrm{By \; inspection \; we \; know \;}  (x=y= -1,1)\\\\$
            Let $x = 0\\$

            \begin{align*}
                f_{x} &= 0 \\
                f_{y} &= 6y^2 - 9 = 0\\
                \therefore y &= \sqrt{\frac{9}{6}} = \sqrt{\frac{3}{2}}\\\\
            \end{align*}

            Let $y = 0\\$

            \begin{align*}
                f_{x} &= -6x^2 = 0 \\
                f_{y} &= 3x^2 -9 
            \end{align*}
            \begin{center}
                No solution for $x$ when $y=0$ \newline
            \end{center}

            The critical points are $\rightarrow [(1,1), \; (-1,-1), \; (0, \sqrt{\frac{3}{2}})]$

        \item Classify the critical point $(0,\sqrt{\frac{3}{2}})$
        
        \begin{align*}
            D &= f_{xx}(0, \sqrt{\frac{3}{2}}) \times f_{yy}(0, \sqrt{\frac{3}{2}}) - f_{xy}^2 (0, \sqrt{\frac{3}{2}})\\\\
            f_{xx}(0, \sqrt{\frac{3}{2}}) &= 3\sqrt{6}\\
            f_{yy}(0, \sqrt{\frac{3}{2}}) &= 6\sqrt{6}\\
            f_{xy}(0, \sqrt{\frac{3}{2}}) &= 0\\\\
            D &= 3\sqrt{6} \times 6\sqrt{6} - 0^2 = 108
        \end{align*}
        Since $D  > 0$ and $f_{xx} > 0$ we know this critical point is a local minimum\\
    
    \end{enumerate}

    \item Quick questions
    
    \begin{enumerate}
        \item Determine the directional derivative of $f(x,y,z) = e^x \cdot \cos(y) \cdot (1-z)^2$ in direction $\bar{u} =  (0.36, 0.48, 0.8)$ from the origin:
        
        \begin{align*}
            D_u f(x_0,y_0,z_0) &= f_x(x_0,y_0,z_0)\bar{u}_1 + f_y(x_0,y_0,z_0)\bar{u}_2 + f_z(x_0,y_0,z_0)\bar{u}_3\\\\
            f_{x} &= e^{x}cos(y)(1-z)^{2} \\
            f_{y} &= -e^{x}sin(y)(1-z)^2 \\
            f_{z} &= e^{x}cos(y)(2z - 2)\\\\
            f_x(0,0,0) &= 1\cdot 1 \cdot1 = 1\\
            f_y(0,0,0) &= -1 \cdot 0 \cdot 1 = 0\\
            f_z(0,0,0) &= 1 \cdot 1 \cdot -2 = -2\\\\
            D_u f(0,0,0) &= 0.36 -1.6 = -1.24\\
        \end{align*}

        \item Determine the local linear approximation of $f(x, y, z) = (1+x)(1-y^2)(1-z)^2$ at the point $(1, 2, 3)$
        
        \begin{align*}
            L(x,y,z) &= f(x_0, y_0, z_0) + f_x(x_0, y_0, z_0)(x-x_0) + f_y(x_0, y_0, z_0)(y-y_0) + f_z(x_0, y_0, z_0)(z-z_0)\\\\
            f_{x}        & = (1-y^2)(1-z)^2                               \\
            f_{y}        & = (1+x)(-2y)(1-z)^2                            \\
            f_{z}        & = 2(1+x)(1-y^2)(z-1)                           \\\\
            f(1,2,3)     & = (1+1)(1-2^2)(1-3)^2=-24                      \\  
            f_{x}(1,2,3) & = (1-2^2)(1-3)^2=-12                           \\
            f_{y}(1,2,3) & = (1+1)(-2(2))(1-3)^2=-32                      \\
            f_{z}(1,2,3) & = 2(1+1)(1-2^2)(3-1)=-24                       \\\\
            L(1,2,3)     & = -24 -12(x - 1) -32(y - 2) -24(z - 3)\\
            L (1,2,3)    & = 124-12x-32y-24z\\
        \end{align*}
    
        \item Determine the 2nd degree Taylor polynomial of $ f(x,y) = e^{-x^2}e^{-y^2} $ as the point (1,1)

        \begin{align*}
            L(x,y) &= f(x_0, y_0) + f_x(x_0, y_0)(x-x_0) + f_y(x_0, y_0)(y-y_0)\\
            p_2(x,y) &= L(x,y) + \frac{1}{2}\left[ f_{xx}(x_0, y_0)(x-x_0)^2 + 2f_{xy}(x_0, y_0)(x-x_0)(y-y_0) + f_{yy}(x_0, y_0)(y-y_0)^2 \right]\\\\
            f_x     &= (-2x) e^{-x^2}e^{-y^2}             \\
            f_y     &= (-2y) e^{-x^2}e^{-y^2}             \\\\
            f_{xx}  &= (4 x^2 - 2) e^{-x^2 - y^2}         \\
            f_{yy}  &= (4 y^2 - 2) e^{-x^2 - y^2}         \\
            f_{xy}  &= (4xy) e^{-x^2}e^{-y^2}             \\\\
            L(1,1)  &= e^{-2} -2e^{-2}(x-1) - 2e^{-2}(y-1) = e^{-2}(5-2x-2y)\\
            p_2(1,1)&= e^{-2}(5-2x-2y) + \frac{1}{2} \left[  2e^{-2}(x-2)^2 + 2e^{-2}(y-2)^2 + 8e^{-2}(x-1)(y-1)\right] \\
                    &= e^{-2}(x^2 + y^2 -8x -8y +4xy +11)
        \end{align*}

        \item Determine the gradient of $f(x,y) = x^3+y^3-4x-2y$ along the curve \\$(x(t), y(t)) = (t^3 - 2t, t^2)$ when $t=1$
        
        \begin{align*}
            \bigtriangledown f(x,y) &= f_x(x,y)\textbf{i} + f_y(x,y)\textbf{j}\\\\
            f_x &= 3x^2 -4\\
            f_y &= 3y^2 -2\\
            \therefore \bigtriangledown f(x,y) &= (3x^2 -4)\textbf{i} + (3y^2 -2)\textbf{j}\\\\
            (x(1),y(1)) &= (-1,1)\\
            \bigtriangledown f(-1,1) &= f_x(-1,1)\textbf{i} + f_y(-1,1)\textbf{j}\\
            &= -1 \textbf{i} + 1 \textbf{j}
        \end{align*}

        \item Determine the tangent plane to the surface $z = x^2 + xy -y^4$ at the point $(x,y)=(2,1)$
        \begin{align*}
            F(x,y,z) &= z - x^2 - xy + y^4 = 0\\\\
            \mathrm{Find \; } &z \mathrm{\; at\; the \; point \;} (x,y) = (2,1)\\
            z &= 2^2 + 2 -1^4=5\\\\
            \bigtriangledown F(x,y,z) &= (-2x-y) \textbf{i} + (4y^3-x)\textbf{j} + \textbf{k}\\
            \bigtriangledown F(2,1,5) &= -5\textbf{i} + 2\textbf{j} + \textbf{k}\\\\
            \mathrm{Tangent Plane} &= -5(x-2) + 2(y-1) + (z-5)\\
            z &= 5x -2y -3
        \end{align*}
    
    \end{enumerate}

    \item Double Integrals
    
    \begin{enumerate}

        \item Determine the integral of $f(x,y) = e^{-x}\cos(y)$ over the rectangular region \\$R = {(x,y) : x \in [0,2], y \in [\frac{-\pi}{2}, \frac{\pi}{2}]}$'
        
        \begin{align*}
            &\int_{-\pi/2}^{\pi/2} \int_{0}^{2} e^{-x}\cos(y) \,dx \,dy\\\\
            &= \int_{-\pi/2}^{\pi/2} \cos(y) \int_{0}^{2} e^{-x} \,dx \,dy\\
            &= \int_{-\pi/2}^{\pi/2} \cos(y) \Big| -e^{-x} \Big|_{x = 0}^{x = 2} \,dy\\
            &= \int_{-\pi/2}^{\pi/2} \cos(y) (-e^{-2} - -e^0) \,dy\\\\
            &= \int_{-\pi/2}^{\pi/2} \cos(y) (-e^{-2} +1) \,dy\\
            &= (-e^{-2} +1) \Big| \sin(y) \Big|_{y = -\pi/2}^{y = \pi/2}\\
            &= (-e^{-2} +1) (\sin(\pi/2) - \sin(-\pi/2))\\\\
            &= -2e^{-2} +2\\
        \end{align*}
    
        \item Determine the integral of $f(x,y)=\sin(x+y)$ over the triangular region for which $ x \geq 0, \; y \geq 0 $ and $ x+y \leq \pi $
        
        \begin{align*}
            &\int_{0}^{\pi} \int_{0}^{\pi - x} \sin(x+y) \,dy  \,dx \\\\
            &= \int_{0}^{\pi} \Big|-\cos(x+y) \Big|_{0}^{\pi - x}  \,dx \\
            &= \int_{0}^{\pi} -\cos(\pi) + cos(x) \,dx \\
            &= \int_{0}^{\pi} 1 + \cos(x) \,dx \\\\
            &= \Big| x +\sin(x) \Big|_{0}^{\pi} \\
            &= (\pi + \sin(\pi)) - (0 +\sin(0)) \\
            &= \pi
        \end{align*}

        \item Determine the area of the region $R = \{(x,y) : e^{y/3} \leq x \leq 10+\sin(y), y \in [0,5]\}$ 
        
        \begin{align*}
            &\int_{0}^{5} \int_{e^{y/3}}^{10+\sin(y)} \,dx  \,dy \\\\
            &\int_{0}^{5} \Big| x  \Big|_{e^{y/3}}^{10+\sin(y)} \,dy \\
            &\int_{0}^{5}  10+\sin(y) - e^{y/3} \,dy \\\\
            &= \Big| 10y -\cos(y) - 3e^{y/3} \Big|_{0}^{5} \\
            &= 50 -\cos(5) - 3e^{5/3} + \cos(0) + 3e^{0} \\
            &= 37.832 \\
        \end{align*}

        \item Determine the average of $f(x,y) = 3y-2x$ over the region \\$R = \{ (x,y) : 0 \leq y \leq 4-x^2, x \in [-2,2] \}$
        
        \begin{align*}
            \mu &= \frac{1}{|R|} \iint_R f(x,y) dA\\\\
            |R| &= \int_{-2}^{2} \int_{0}^{4-x^2} \,dy  \,dx \\
                &= \int_{-2}^{2} \big| y \Big|_{0}^{4-x^2} \,dx \\
                &= \int_{-2}^{2} 4-x^2 \,dx \\\\
                &= \Big| 4x - \frac{x^3}{3} \Big|_{-2}^{2} \\
                &= 8 - \frac{2^3}{3} + 8 - \frac{2^3}{3} \\ 
                &= \frac{32}{3}\\\\
                &\int_{-2}^{2} \int_{0}^{4-x^2} 3y - 2x \,dy  \,dx \\
                &= \int_{-2}^{2} \Big| \frac{3y^2}{2} - 2xy \Big|_{0}^{4-x^2} \,dx \\
                &= \int_{-2}^{2} \frac{3(4-x^2)^2}{2} - 2x(4-x^2) \,dx \\
                &= \int_{-2}^{2} \frac{3x^4}{2} + 2x^3 - 12x^2 - 8x + 24 \,dx \\\\
                &= \int_{-2}^{2} \frac{3x^5}{10} + \frac{x^4}{2} - 4x^3 - 4x^2 + 24x  \\
                &= \Big| \frac{3x^5}{10} + \frac{x^4}{2} - 4x^3 - 4x^2 + 24x \Big|_{-2}^{2} \\
                &= (\frac{3(2)^5}{10} + \frac{(2)^4}{2} - 4(2)^3 - 4(2)^2 + 24(2)) - (\frac{3(-2)^5}{10} + \frac{(-2)^4}{2} - 4(-2)^3 - 4(-2)^2 + 24(-2))\\
                &= \frac{256}{5}\\\\
            \mu &= \frac{\frac{256}{5}}{\frac{32}{3}} \\
                &= \frac{24}{5} = 4.8
        \end{align*}

        \item Determine the surface area of the surface described by $z = \sqrt{9 - x^2}$ over the region $R = \{ (x,y) : 0 \leq y \leq x, x \in [0,3] \}$ 
        
        \begin{align*}
            \mathrm{Surface \; Area \;} &= \iint_R \sqrt{ f_x(x,y)^2 + f_y(x,y)^2 +1} dA\\\\
            \mathrm{Integrand} &= \sqrt{ f_x(x,y)^2 + f_y(x,y)^2 +1}\\\\
            f_x &= -\frac{x}{\sqrt{9 - x^2}}\\
            f_y &= 0 \\\\
            \mathrm{Integrand} &= \sqrt{ \left(-\frac{x}{\sqrt{9 - x^2}}\right)^2 +1}\\\\
            &= \sqrt{ \frac{x^2}{9-x^2} +1} = \frac{3}{\sqrt{9-x^2}}\\\\
            \mathrm{Surface \; Area \;} &= \int_{0}^{3} \int_{0}^{x} \frac{3}{\sqrt{9-x^2}} \,dy  \,dx \\ 
            &= \int_{0}^{3} \Big| \left( \frac{3}{\sqrt{9-x^2}} \right) y \Big|_{0}^{x}  \,dx \\
            &= \int_{0}^{3} \left( \frac{3}{\sqrt{9-x^2}} \right) x  \,dx \\\\
            &= \Big| -3\sqrt{9-x^2} \Big|_{0}^{3}\\
            &= -3\sqrt{0} + 3\sqrt{9}\\
            &= 9
        \end{align*}

    \end{enumerate}

\end{enumerate}
\end{preview}
\end{document}