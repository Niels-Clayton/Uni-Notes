\documentclass[a4paper,11pt, twocolumn]{article}
\usepackage[left=2.5cm, right=2.5cm, top=1.5cm, bottom=1.5cm]{geometry}
\usepackage{graphicx}
\usepackage{amssymb}
\usepackage{amsmath}
\usepackage[procnames]{listings}
\usepackage{xcolor}
\usepackage{hyperref}

\hypersetup{ %color attributes of citation, link, etc.
    colorlinks=true,
    linkcolor=blue,
    filecolor=gray,
    urlcolor=blue,
    citecolor=blue,
}

\setlength{\parindent}{0pt}

\newcommand{\matlab}{\textsc{Matlab}} %very important and totally necessary addition
\newcommand{\parallelsum}{\mathbin{\!/\mkern-5mu/\!}}

%'codify' text for snippets
\usepackage{xcolor}
\definecolor{codegray}{gray}{1}
\newcommand{\code}[1]{\colorbox{codegray}{\texttt{#1}}}

\definecolor{keywords}{RGB}{255,0,90}
\definecolor{comments}{RGB}{0,0,113}
\definecolor{p_red}{RGB}{160,0,0}
\definecolor{p_green}{RGB}{0,150,0} 
\lstset{language=Python, 
        basicstyle=\ttfamily\small, 
        keywordstyle=\color{keywords},
        commentstyle=\color{comments},
        stringstyle=\color{p_red},
        showstringspaces=false,
        identifierstyle=\color{p_green},
		procnamekeys={def,class}}

\graphicspath{ {./images/} }
           
\begin{document}
\title{\LARGE{\textbf{ECEN425 Assignemnt 1}\\Going Beyond Datasheets}}
\author{Niels Clayton : 300437590}
\date{}
\maketitle

\section{Battery Technologies} 

\subsection{Lead Acid Batteries}

Lead acid batteries are an inexpensive rechargeable battery technology. They feature the lowest self discharge rate amongst rechargeable batteries \cite{lead_acid}, as well as a high specific power and good low and high temperature performance. They do however have a long charging time compared to other batteries, between 8 to 16 hours. They are made from generally considered highly toxic materials however their ease of recycling means that they have a 99\% recycling rate. 

\subsubsection{Storage}

Lead acid batteries must be stored in a state of full charge. Prolonged periods of low charge cause sulfation to occur, permanently shortening the life of the battery \cite{sulfation}. They must also be stored within their operating temperature range, low temperatures will cause them to freeze and high temperatures will lead to a loss of the electrolyte, both of which will damage the battery.

\subsubsection{Battery Maintenance \& Safety}

\textbf{Flooded Lead Acid}\\
To maintain a flooded lead acid battery, the lead plates must never become exposed. To prevent this, regular checks of the batteries water level should be conducted, and watering of the battery must take place. The battery must be topped up with distilled water to the indicated level, but only after a full charge to prevent an overflow.
It is important to know that the charging of a flooded lead acid battery can lead to the production of both hydrogen and oxygen gas. To combat this, the battery must always be charged in a well ventilated area. \\

\textbf{Sealed Lead Acid}\\
Sealed lead acid batteries come with a set amount of electrolyte that is non-replaceable. They are equipped with a valve that will vent gas that is produced, making them very low maintenance. This means however that incorrect usage such as overcharging or storing at high temperatures will degrade the performance of the battery permanently. Sealed lead acid batteries must be charged at a lower voltage than flooded to avoid the gas generation stage of the charge cycle. They must also be kept within their optimum operating temperature, as every 8$^o$C above this temperature threshold cuts the battery life in half \cite{lead_acid}.

\subsubsection{Battery Applications}

Lead acid batteries have a low specific energy when compared to other battery Technologies. Because of this they tend to be used in situations where the weight of the battery is not the limiting factor.

\textbf{\\Deep cycle Lead Acid}\\
The deep cycle lead acid battery is built for maximum capacity and a reasonably high cycle count. This is done by increasing the thickness of the lead plates inside of the battery. Although this battery is designed for cycling, fully discharging the battery will still stress it. This means that the cycle count is still dependant on the depth of discharge. This battery is used in continuous power applications, these include uninterruptible power supply’s (UPS’s), and external battery storages for off grid housing or emergencies.

\textbf{\\Starter Lead Acid}\\
The starter lead acid battery is built for momentary high power, high current loads lasting only a few seconds. This is done by making the lead plates thin and porous to increase their surface area. This gives the battery a very low internal resistance, however it also greatly decreases the number of deep cycles the battery can withstand. Starter batteries are categorised by their cold cranking amps (CCA), which defines the current the battery can deliver at cold temperatures. This battery is often used to crank engines, but is suitable for any task requiring occasional high current loads.

\subsection{Nickel-cadmium - NiCd}

Nickel-cadmium (NiCd) batteries are an easy to maintain and long lasting battery. They feature a high cycle count when compared to most other rechargeable batteries, as well as being generally rouged to a wide range of conditions. NiCd batteries are constructed from toxic materials, because of this they can be hard to purchase, and require specific care when being disposed.

\subsubsection{Storage}

NiCd Batteries are capable of being stored at any state of charge, however it is best to store them in a discharged state, preferably between 40\% and 0\% charge. NiCd batteries should also not be stored for extended periods of time without use (over a year), this can lead to internal shorting of the battery due to the formation of dendrites (thin, conductive crystals) that will permanently destroy the battery\cite{dendrites}. 

\subsubsection{Battery Maintenance \& Safety}

NiCd Batteries are one of the few batteries that will benefit from a full discharge of the cells. When using NiCd batteries you should aim to perform a full discharge every 1-3 months to prevent the build up of crystals within the battery. Never short circuit the battery to discharge it, this will produce hydrogen gas that can lead to explosions. 
NiCd batteries also require the most complex circuitry for charging, and should never be overcharged as this will damage the battery. 
It is advised that after storage a NiCd battery should be primed before use by trickle charging the battery for 16-24 hours.

\subsubsection{Battery Applications}

NiCd Batteries are commonly used in power tools, medical tools, and almost exclusively used in the aviation industry. This is due to the batteries great cycling ability, near constant voltage at most charge states, and wide operating temperature range. NiCd Batteries are also a very rugged, and will not be damaged by a wide range of conditions. It should be noted that NiCd Batteries are currently being phased out of commercial use due to their toxicity, causing them to be hard to purchase and dispose of.  

\subsection{Nickel-metal-hydride - NiMH}

Nickel-metal-hydride (NiMH) batteries are a commonly available rechargeable battery for consumer use and are a seen as a cleaner replacement for the more toxic NiCd batteries . NiMH batteries feature a much higher specific energy than NiCd batteries, while maintaining the same robustness. They however have a much lower cycle capability, and feature a very high self discharge rate, making them less ideal for applications of intermittent, small scale usage. 

\subsubsection{Storage}

NiMH is also capable of being stored in any state of charge, however it is preferable to store them between 40\% and 0\% charge. Due to the high self discharge rate of NiMH batteries, it is important to perform a charge and discharge cycle at least once every year to preserve the battery, and preferably once every couple months. When storing NiMH batteries, ensure that they are not subjected to extreme temperatures.  

\subsubsection{Battery Maintenance \& Safety}

NiMH batteries are very sensitive to being overcharged, and have very complex charging circuitry to protect them. Because of this it is important to use a charger that is capable of detecting the different NiMH charge stages, and correctly charge the battery.

\subsubsection{Battery Applications}

NiMH batteries are commonly used in short term high drain situations. This has lead to them being commonly used in hybrid and electric cars, and a large selection of consumer applications such as cameras. They are relatively low maintenance and cost, and their ease of storage makes them very consumer friendly. They should not be used in long term low power situations as their self discharge can cause them to lose up to 20\% charge in the first 20 hours, and 10\% each following month. 

\subsection{Lithium-ion - Li-Ion}

The key to the superior specific energy is the high cell voltage of 3.60V\\
Li-ion is a low-maintenance battery, The battery has no memory and does not need exercising (deliberate full discharge) to keep it in good shape.

\subsubsection{Storage}

\subsubsection{Battery Maintenance \& Safety}

\subsubsection{Battery Applications}


\subsection{Lithium-polymer - Li-Po}

\subsubsection{Storage}

\subsubsection{Battery Maintenance \& Safety}

\subsubsection{Battery Applications}


\newpage
\onecolumn
\bibliographystyle{IEEEtran}
\bibliography{ref}

\end{document}

% Insert image
\begin{center}
    \fbox{\includegraphics[width=0.9\textwidth]{image_name.png}}
\end{center}