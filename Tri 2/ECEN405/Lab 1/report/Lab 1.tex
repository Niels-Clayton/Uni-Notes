\documentclass[a4paper,11pt]{article}
\usepackage[left=2.5cm, right=2.5cm, top=1.5cm, bottom=1.5cm]{geometry}
\usepackage{graphicx}
\usepackage{amssymb}
\usepackage{amsmath}
\usepackage{xcolor}
\usepackage[active,tightpage]{preview}
\usepackage{hyperref}
\usepackage{pythonhighlight}

\hypersetup{ %color attributes of citation, link, etc.
    colorlinks=true,
    linkcolor=blue,
    filecolor=gray,
    urlcolor=blue,
    citecolor=blue,
}

\setlength{\parindent}{0pt}

\renewcommand{\PreviewBorder}{1in}
\newcommand{\Newpage}{\end{preview}\begin{preview}}
\newcommand{\matlab}{\textsc{Matlab}} %very important and totally necessary addition
\newcommand{\parallelsum}{\mathbin{\!/\mkern-5mu/\!}}

\newcommand\Item[1][]{%
  \ifx\relax#1\relax  \item \else \item[#1] \fi
  \abovedisplayskip=0pt\abovedisplayshortskip=0pt~\vspace*{-\baselineskip}}

%'codify' text for snippets
\usepackage{xcolor}
\definecolor{codegray}{gray}{1}
\newcommand{\code}[1]{\colorbox{codegray}{\texttt{#1}}}


\graphicspath{ {../images/} }
           
\begin{document}
\begin{preview}
\title{\LARGE{\textbf{ECEN405 Lab 1 Report\\Pulse Width Modulation}}}
\author{Niels Clayton : 300437590\\\textbf{Lab Partner:} Nickolai Wolfe}
\date{}
\maketitle
\hrule

\begin{enumerate}
    \item 
    $C_1 = 10\;nF$\\
    
    \item 
    $f_{min} = 297$ Hz\\$f_{max} = 3.03$ MHz\\
    
    \item 
    Schematic of the PWM generator
    \begin{center}
        \includegraphics[width = \textwidth]{schematic_1.png}
    \end{center}
    \vspace{20pt}

    \item 
    Conduction losses: $$P_{cond} = 0.6\;W $$
    Minimum frequency switching losses $$P_{sw}= 78.4 \;\mu W$$
    Maximum frequency switching losses $$P_{sw}= 0.799 \; W$$\\
    
    \item 
    Actual photo capturing the look of approval on Danny B's face as I complete the circuit and proudly show him.
    \begin{center}
        \includegraphics[width = \textwidth]{Danny_B.jpg}
    \end{center}
    \vspace{20pt}

    \item 
    50\% duty cycle PWM
    \begin{center}
        \includegraphics[width = 0.8\textwidth]{50_duty.JPG}
    \end{center}
    \vspace{10pt}

    10\% duty cycle PWM
    
    \begin{center}
        \includegraphics[width = 0.8\textwidth]{10_duty.JPG}
    \end{center}
    \vspace{10pt}

    90\% duty cycle PWM
    
    \begin{center}
        \includegraphics[width = 0.8\textwidth]{90_duty.JPG}
    \end{center}
    \vspace{10pt}
    
    
    \item It was noted that the theoretical minimum and maximum achieved switching frequencies of the circuit were not achieved by the designed circuit. Although it is hard to exactly pinpoint where this issue comes from, there are a range of possible factors that would affect these frequencies. By creating the circuit on a breadboard, we introduce parasitic capacitances, and inductances to the circuit, which will affect the operation of the op-amps. It is also possible that op-amp characteristics such as slew rate are affecting the frequency of the output. \\
    
    
    \item To get both an inverted and non-inverted signal, you can switch the input terminals of the comparator. Since the LM319 is a dual comparator, it is possible to have both the inverted and non-inverted signals out from the same IC. \\
    

    \item 

    Below is a block diagram of the two sub-circuits that can be found with the triangle wave generator, and their respective outputs. The first circuit is a square wave generator or oscillator, and the second is an integrator. 
    
    This circuit is made using a gain amplifying opamp that has positive feedback and hysteresis. This positive feedback causes it to become unstable, and the hysteresis causes it to oscillate from rail to rail. The frequency of this oscillation can be set using R3. 

    The second sub-circuit is an opamp integrator. This sub-circuit will integrate the square wave generated by the first to produce a triangle wave. The scaling of this circuits output (both frequency and gain) is controlled using R4. 
    
    \begin{center}
        \includegraphics[width = 1\textwidth]{block_diagram.png}
    \end{center}
    \vspace{10pt}


\end{enumerate}

\section*{Appendix}

\textbf{Q1:} Equation to calculate the capacitor size
$$ C_{1}=\frac{R_{2}+R_{3}}{4R_{1}\left(R_{4}+R_{5}\right)F_{T}} $$\\

\textbf{Q2:} Equations to calculate the minimum and maximum frequencies
$$ f_{min}=\frac{R_{2}}{4R_{1}\left(R_{4}+R_{5}\right)C_{1}} = 297\;Hz $$ \\
$$ f_{max}=\frac{R_{2}+R_{3}}{4R_{1}\cdot R_{5}\cdot C_{1}} = 3.03\;MHz $$\\

\textbf{Q4:} \\\\
Conduction switching losses:
$$ P_{cond} = R_{DS(on)} \cdot d \cdot I^2 = 0.6\;W $$
Minimum frequency switching losses:
$$ P_{sw} = \frac{1}{2}V_{in} \cdot I_o (t_{c(on)} + t_{c(off)})f_{min} = 78.4 \;\mu W $$
Maximum frequency switching losses:
$$ P_{sw} = \frac{1}{2}V_{in} \cdot I_o (t_{c(on)} + t_{c(off)})f_{max} = 0.799 \; W $$


\end{preview}
\end{document}